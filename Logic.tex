\documentclass[a4paper,oneside,11pt,DIV=12,parskip=half]{scrartcl}
\usepackage[english]{babel}
\usepackage[utf8]{inputenc}
\usepackage[T1]{fontenc}
\usepackage{microtype}
\usepackage{lmodern}
\usepackage{amsmath}
\usepackage{amssymb}
\usepackage{showkeys}
\usepackage{amsthm}
\usepackage{booktabs}
\usepackage{graphicx}
\usepackage{listings}
\usepackage{enumerate}
\usepackage[usenames, dvipsnames]{color}
\usepackage{hyperref}

\title{Logic}
\author{ Luka Ili\'{c}}


\newcommand{\R}{\mathbb R}
\newcommand{\N}{\mathbb N}
\newcommand{\E}{\mathcal E}
\newcommand{\s}{\mathcal S}

\newcommand{\pmat}[1]{\begin{pmatrix}
		#1
\end{pmatrix}}
\newcommand{\abs}[1]{\left| #1\right|}

\newcommand{\skal}[1]{\left \langle #1 \right\rangle}


\DeclareMathOperator{\grad}{grad}

\theoremstyle{plain}
\newtheorem{theorem}{Satz}[section]
\newtheorem{lemma}[theorem]{Lemma.} 
\newtheorem{proposition}[theorem]{Proposition.}  
\newtheorem{corollary}[theorem]{Korollar.}

\theoremstyle{definition}
\newtheorem{definition}[theorem]{Definition.}
\newtheorem{remark, definition}[theorem]{Bemerkung und Definition.}
\newtheorem{lemma, definition}[theorem]{Lemma und Definition.}  
\newtheorem{theorem, definition}[theorem]{Satz und Definition.}  

\theoremstyle{remark}
\newtheorem*{remark}{\textbf{Bemerkung}}
\newtheorem*{exercise}{\textbf{Aufgabe}}
\newtheorem*{example}{\textbf{Beispiel}}
\newtheorem*{remark, example}{\textbf{Bemerkung und Beispiel}} 


\begin{document}

\maketitle

\pagebreak

\tableofcontents

\pagebreak

\section{Propositional logic}

Convention: In this course we write $T$ for true and $F$ for false.

\begin{definition}
	The language of propositional logic consists of following symbols:
 \emph{propositional variables} denoted (mostly) by $ p,q, \dots $ or 	$p_1,p_2,\dots, q_1, q_2, \dots$
	and the \emph{connectives} $\land, \lor, \lnot, \rightarrow, \leftrightarrow$.
\end{definition}

\begin{definition}
	A \emph{propositional formula} is a string of symbols obatained in the following way
\begin{enumerate}
\item Any variable is a formula \\
\item If $\phi $ and $\psi$ are formulas then so are
$(\phi \land \psi),(\phi \lor \psi), (\lnot \phi), (\phi \rightarrow \psi), (\phi \leftrightarrow \psi)$
\item Any formula is obtained in this way.
\end{enumerate}
\end{definition}

\begin{definition}
A \emph{truth function} of $n$ variables is a function 
	\[ f: \{ T,F \}^n \rightarrow \{ T,F \} \quad. \]
\end{definition}

Exercize: How many functions are there for $n$ variables?

\begin{definition}
Suppose $\phi$ is a formula with variables $p_1, \dots, p_n$ then we obtain a truth function $F_{\phi}: \{T,F\}^n \rightarrow \{T,F\}$ whose value at $(x_1, \dots, x_n) \quad x_i \in \{T,F\} $ is the truth value of $\phi$ when $p_i$ has value $x_i$. THe function $F_{\phi}$ is the \emph{truth function of $\phi$}.
\end{definition}

\end{document}