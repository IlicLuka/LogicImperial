\documentclass[a4paper,oneside,11pt,DIV=12,parskip=half]{scrartcl}
\usepackage[english]{babel}
\usepackage[utf8]{inputenc}
\usepackage[T1]{fontenc}
\usepackage{microtype}
\usepackage{lmodern}
\usepackage{amsmath}
\usepackage{amssymb}
\usepackage{showkeys}
\usepackage{amsthm}
\usepackage{booktabs}
\usepackage{graphicx}
\usepackage{listings}
\usepackage{enumerate}
\usepackage[usenames, dvipsnames]{color}
\usepackage{hyperref}

\title{Logic}
\author{ Luka Ili\'{c}}


\newcommand{\R}{\mathbb R}
\newcommand{\N}{\mathbb N}
\newcommand{\E}{\mathcal E}
\newcommand{\s}{\mathcal S}

\newcommand{\pmat}[1]{\begin{pmatrix}
		#1
\end{pmatrix}}
\newcommand{\abs}[1]{\left| #1\right|}

\newcommand{\skal}[1]{\left \langle #1 \right\rangle}


\DeclareMathOperator{\grad}{grad}

\theoremstyle{plain}
\newtheorem{theorem}{Theorem.}[section]
\newtheorem{lemma}[theorem]{Lemma.} 
\newtheorem{proposition}[theorem]{Proposition.}  
\newtheorem{corollary}[theorem]{Corollary.}

\theoremstyle{definition}
\newtheorem{definition}[theorem]{Definition.}
\newtheorem{remark, definition}[theorem]{Remark and Definition.}
\newtheorem{lemma, definition}[theorem]{Lemma and Definition.}  
\newtheorem{theorem, definition}[theorem]{Theorem and Definition.}  

\theoremstyle{remark}
\newtheorem*{remark}{\textbf{Remark}}
\newtheorem*{exercise}{\textbf{Exercise}}
\newtheorem*{example}{\textbf{Example}}
\newtheorem*{remark, example}{\textbf{Remark and Exercise}} 


\begin{document}

\maketitle

\pagebreak

\tableofcontents

\pagebreak

\section{Propositional logic}

Convention: In this course we write $T$ for true and $F$ for false.

\begin{definition}
	The language of propositional logic consists of following symbols:
 \emph{propositional variables} denoted (mostly) by $ p,q, \dots $ or 	$p_1,p_2,\dots, q_1, q_2, \dots$
	and the \emph{connectives} $\land, \lor, \lnot, \rightarrow, \leftrightarrow$.
\end{definition}

\begin{definition}
	A \emph{propositional formula} is a string of symbols obatained in the following way
\begin{enumerate}
\item Any variable is a formula \\
\item If $\phi $ and $\psi$ are formulas then so are
$(\phi \land \psi),(\phi \lor \psi), (\lnot \phi), (\phi \rightarrow \psi), (\phi \leftrightarrow \psi)$
\item Any formula is obtained in this way.
\end{enumerate}
\end{definition}

\begin{definition}
A \emph{truth function} of $n$ variables is a function 
	\[ f: \{ T,F \}^n \rightarrow \{ T,F \} \quad. \]
\end{definition}

\begin{exercise} 
How many functions are there for $n$ variables?

\end{exercise}

\begin{definition}
Suppose $\phi$ is a formula with variables $p_1, \dots, p_n$ then we obtain a truth function $F_{\phi}: \{T,F\}^n \rightarrow \{T,F\}$ whose value at $(x_1, \dots, x_n) \quad x_i \in \{T,F\} $ is the truth value of $\phi$ when $p_i$ has value $x_i$. THe function $F_{\phi}$ is the \emph{truth function of $\phi$}.
\end{definition}

\begin{example}
What is the truth function of 
\[ (((p \rightarrow q) \land (q \rightarrow (\lnot p))) \rightarrow (\lnot p)) \quad ? \]

It is always true.
\end{example}

\begin{definition}
A propositional formula $\phi$ whose truth function $F_{\phi}$ is always true is called \emph{tautology}.
Say that formulas $\phi, \psi$ are \emph{logically equivalent} (l.e.) if they have the same truth function.
\end{definition}

\begin{remark}
$\phi,\psi$ are l.e. iff $(\phi \leftrightarrow \psi)$ is a tautology.
Also, suppose that we got some formula $\phi$ with variables $p_1, \dots, p_n$ and $\phi_1,\dots,\phi_n$ are formulas with variables $q_1, \dots, q_r$.
For each $i \leq n$ substitute $\phi$ in place of $p_i$ in $\phi$. Then the result is a formula $ \psi$ and if $\phi$ is a tautology, then so is $\psi$.
\end{remark}

\begin{proof}
The first statement is easy. For the second remark that
	\[ F_{\psi}(q_1, \dots, q_r) = F_{\phi}(F_{\phi_1}(q_1, \dots, q_r), \dots, F_{\phi_n}(q_1, \dots, q_r)) \]
by the induction on the number of connectives in $\phi$.
\end{proof}

\begin{example}
\begin{enumerate}
\item $(p_1 \land ( p_2 \land p_3))$ is l.e. to $((p_1 \land p_2) \land p_3)$,
\item same with $ \lor $,
\item $(p_1 \lor (p_2 \land p_3))$ is l.e. to $((p_1 \lor p_2) \land (p_1 \lor p_3))$ 
\item similar the other way around.
\item etc.

\end{enumerate}
\end{example}

\begin{remark}
Note that by the remark above, we can boost these equivalences by substituting formulas for the variables.
\end{remark}

\begin{definition}

Say that a set of connectives is \emph{adequade} if for eery $n \geq 1$, everz truth function of $n$ variables is the truth function of some formula which isnvolves onlz connectives from the set and variables $p_1, \dots, p_n$.
\end{definition}

\begin{theorem}
The set $\{ \lnot, \land, \lor\}$ is adequate. 
\end{theorem}

\begin{proof}
Let $G: \{T,F\}^n \rightarrow \{T,F\}$
\begin{enumerate}
\item $G(v) = F$ for all $v \in \{T,F\}$.
Take $\phi$ to be $(p_1 \land(\lnot p_1))$ then $ G = F_{\phi}$
\item (\emph{Disjunctive Normal Form}
List the $v \in \{ T,F \}^n$ with $G(v) = T$ as $v_1, \dots, v_r$.
Write $v_i = (v_{i1},\dots,v_{in})$
Define
\[ q_{ij} = \begin{cases} p_j \text{if} v_{ij} = T \\  ( \lnot p_j) \text{if} v_{ij} = F
\end{cases}
So $q_{ij}$ has value $T$ iff $p_j$ has value $v_{ij}$.
Let $\psi_i$ be
	\[ (q_{i1}, \dots, q_{in})\]
Then $F_{\psi_i}(v) = T$ iff each $q_{ij}$ has value $T$ iff $v = v_i$.

Let $\theta$ be $(\phi_1 \lor, \dots , \lor \phi_r)$.
Then $F_{\theta}(v) = T$ iff $F_{\psi_i}(v) = T$ for some $i$ which is equivalent to $v = v_i$ for some $i \leq r$.
Thus $F_{\theta} (v) = T$ iff $G(v) = T$ i.e. $F_{\theta} = G$.
As $\theta$ was constructed using only $\lnot,\lor,\land$ the statement follows.
\end{enumerate}

\end{proof}

\begin{corollary} Suppose $\chi\ is a formula 
Then $\chi$ is l.e. to a formula in dnf.
\end{corollary}




\end{document}