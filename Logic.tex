\documentclass[a4paper,oneside,11pt,DIV=12,parskip=half]{scrartcl}
\usepackage[english]{babel}
\usepackage[utf8]{inputenc}
\usepackage[T1]{fontenc}
\usepackage{microtype}
\usepackage{lmodern}
\usepackage{amsmath}
\usepackage{amssymb}
%\usepackage{showkeys}
\usepackage{amsthm}
\usepackage{booktabs}
\usepackage{graphicx}
\usepackage{listings}
\usepackage{enumerate}
\usepackage[usenames, dvipsnames]{color}
\usepackage{hyperref}

\title{Logic}
\author{ Luka Ili\'{c}}


\newcommand{\R}{\mathbb R}
\newcommand{\N}{\mathbb N}
\newcommand{\E}{\mathcal E}
\newcommand{\s}{\mathcal S}

\newcommand{\pmat}[1]{\begin{pmatrix}
		#1
\end{pmatrix}}
\newcommand{\abs}[1]{\left| #1\right|}

\newcommand{\skal}[1]{\left \langle #1 \right\rangle}


\DeclareMathOperator{\grad}{grad}

\theoremstyle{plain}
\newtheorem{theorem}{Theorem.}[section]
\newtheorem{lemma}[theorem]{Lemma.} 
\newtheorem{proposition}[theorem]{Proposition.}  
\newtheorem{corollary}[theorem]{Corollary.}

\theoremstyle{definition}
\newtheorem{definition}[theorem]{Definition.}
\newtheorem{remark, definition}[theorem]{Remark and Definition.}
\newtheorem{lemma, definition}[theorem]{Lemma and Definition.}  
\newtheorem{theorem, definition}[theorem]{Theorem and Definition.}  

\theoremstyle{remark}
\newtheorem*{remark}{\textbf{Remark}}
\newtheorem*{exercise}{\textbf{Exercise}}
\newtheorem*{example}{\textbf{Example}}
\newtheorem*{remark, example}{\textbf{Remark and Exercise}} 


\begin{document}

\maketitle

\pagebreak

\tableofcontents

\pagebreak

\section*{Preface}
The following notes, are to be regarded as such -- notes. They should contain most of what is written down in the logic lecture at Imperial College London (2018) by professor Evans. More likely then not, there will be a considerable amount of spelling errors (--please report everything to email down below or in Github comments--) that hopefully do not alter any important meaning. These notes will be constantly reread (by you the readers as well as myself) so I hope that at the end of the term most errors will be corrected so that anybody reading this will find good lecture notes for the exam.

\textbf{Anybody willing to help me, can write me an email at \\
\\ luka.ilic18@imperial.ac.uk} \\ \\ Help will only consist of being able to edit errors yourself. (So no need for any texing, except if you really want to.)

This is a project for my fellow students, so I hope it will be appreciated and used. I wish everbody reading this a lot of fun with the following content.

\pagebreak

\section{Propositional logic}

\subsection{Truth functions}
Convention: In this course we write $T$ for true and $F$ for false.

\begin{definition}
	The alphabet of propositional logic consists of following symbols:
 \emph{propositional variables} denoted (mostly) by $ p,q, \dots $ or 	$p_1,p_2,\dots, q_1, q_2, \dots$
	and the \emph{connectives} $\land, \lor, \lnot, \rightarrow, \leftrightarrow$.
\end{definition}

\begin{definition}\label{Def:formula}
	A \emph{propositional formula} is a string of symbols obtained in the following way:
\begin{enumerate}
\item Any variable is a formula. \\
\item If $\phi $ and $\psi$ are formulas then so are
$(\phi \land \psi),(\phi \lor \psi), (\lnot \phi), (\phi \rightarrow \psi), (\phi \leftrightarrow \psi)$.
\item Any formula is obtained in this way.
\end{enumerate}
\end{definition}

\begin{definition}
A \emph{truth function} of $n$ variables is a function 
	\[ f: \{ T,F \}^n \rightarrow \{ T,F \} \quad. \]
\end{definition}

\begin{exercise} 
How many functions are there for $n$ variables?

\end{exercise}

\begin{definition}
Suppose $\phi$ is a formula with variables $p_1, \dots, p_n$ then we obtain a truth function $F_{\phi}: \{T,F\}^n \rightarrow \{T,F\}$ whose value at $(x_1, \dots, x_n) \quad x_i \in \{T,F\} $ is the truth value of $\phi$ when $p_i$ has value $x_i$. The function $F_{\phi}$ is the \emph{truth function of $\phi$}.
\end{definition}

\begin{example}
What is the truth function of 
\[ (((p \rightarrow q) \land (q \rightarrow (\lnot p))) \rightarrow (\lnot p)) \quad ? \]

\end{example}

\begin{definition}
A propositional formula $\phi$ whose truth function $F_{\phi}$ is always true is called \emph{tautology}.
Say that formulas $\phi, \psi$ are \emph{logically equivalent} (l.e.) if they have the same truth function.
\end{definition}

\begin{remark}
$\phi,\psi$ are l.e. iff $(\phi \leftrightarrow \psi)$ is a tautology.
Also, suppose that we got some formula $\phi$ with variables $p_1, \dots, p_n$ and $\phi_1,\dots,\phi_n$ are formulas with variables $q_1, \dots, q_r$.
For each $i \leq n$ substitute $\phi$ in place of $p_i$ in $\phi$. Then the result is a formula $ \psi$ and if $\phi$ is a tautology, then so is $\psi$.
\end{remark}

\begin{proof}
The first statement is easy. For the second remark that
	\[ F_{\psi}(q_1, \dots, q_r) = F_{\phi}(F_{\phi_1}(q_1, \dots, q_r), \dots, F_{\phi_n}(q_1, \dots, q_r)) \]
by induction on the number of connectives in $\phi$.
\end{proof}

\begin{example}
\begin{enumerate}
\item $(p_1 \land ( p_2 \land p_3))$ is l.e. to $((p_1 \land p_2) \land p_3)$,
\item same with $ \lor $,
\item $(p_1 \lor (p_2 \land p_3))$ is l.e. to $((p_1 \lor p_2) \land (p_1 \lor p_3))$ 
\item similar the other way around.
\item etc.

\end{enumerate}
\end{example}

\begin{remark}
Note that by the remark above, we can boost these equivalences by substituting formulas for the variables.
\end{remark}

\begin{definition}

Say that a set of connectives is \emph{adequate} if for evry $n \geq 1$, every truth function of $n$ variables is the truth function of some formula which involves only connectives from the set and variables $p_1, \dots, p_n$.
\end{definition}

\begin{theorem}\label{Th:connectives}
The set $\{ \lnot, \land, \lor\}$ is adequate. 
\end{theorem}

\begin{proof}
Let $G: \{T,F\}^n \rightarrow \{T,F\}$
\begin{enumerate}
\item $G(v) = F$ for all $v \in \{T,F\}$.
Take $\phi$ to be $(p_1 \land(\lnot p_1))$ then $ G = F_{\phi}$
\item (\emph{Disjunctive Normal Form}
List the $v \in \{ T,F \}^n$ with $G(v) = T$ as $v_1, \dots, v_r$.
Write $v_i = (v_{i1},\dots,v_{in})$
Define
\[ q_{ij} = \begin{cases} p_j &\text{if } v_{ij} = T \\  ( \lnot p_j) &\text{if } v_{ij} = F
\end{cases}\]
So $q_{ij}$ has value $T$ iff $p_j$ has value $v_{ij}$.
Let $\psi_i$ be
	\[ (q_{i1}, \dots, q_{in})\]
Then $F_{\psi_i}(v) = T$ iff each $q_{ij}$ has value $T$ iff $v = v_i$.

Let $\theta$ be $(\phi_1 \lor, \dots , \lor \phi_r)$.
Then $F_{\theta}(v) = T$ iff $F_{\psi_i}(v) = T$ for some $i$ which is equivalent to $v = v_i$ for some $i \leq r$.
Thus $F_{\theta} (v) = T$ iff $G(v) = T$ i.e. $F_{\theta} = G$.
As $\theta$ was constructed using only $\lnot,\lor,\land$ the statement follows.
\end{enumerate}

\end{proof}

\begin{definition}
A formula in the form as $\theta$ in the proof above (\ref{Th:connectives}) is said to be in \emph{disjunctive normal form (dnf)}.
\end{definition}

\begin{remark}

Apart from the very intuitive and useful dnf, having a small adequate set at our disposal is useful for the following reason. It shortens induction proofs over the structure of formulas by a considerable amount, as the reader will surely experience in due time.

\end{remark}

\begin{corollary} Suppose $\chi$ is a formula which truth function is not always false. Then $\chi$ is l.e. to a formula in dnf.
\end{corollary}

\begin{proof}
Take $G = F_{\chi}$ and apply the second case from the proof above.
\end{proof}
\begin{example} For
\[ \chi: \quad (( p_1 \rightarrow p_2) \rightarrow (\lnot p_2)) \] 
the truth function $F_{\chi}(v)$ is true precisely when $v = \{T,F\}$ or $ v = \{F, F \}$.
Hence the dnf is:
\[ ((p_1 \land (\lnot p_2)) \lor ((\lnot p_1) \land ( \lnot p_2))). \]

\end{example}

\begin{corollary}
The following sets of connectives are adequate:
\begin{enumerate}
    \item $\lnot, \lor$
    \item $\lnot, \land$
    \item $\lnot, \rightarrow$.
\end{enumerate}
\end{corollary}

\begin{proof}
\begin{enumerate}
    \item By \ref{Th:connectives} we just need to show, that $\land$ can be expressed using $\lnot,\lor$. $(p \land q)$ is l.e. to $(\lnot((\lnot p) \lor (\lnot q))$.
    \item similar to the approach above. $(p \lor q)$ is l.e. to $(\lnot((\lnot p) \land (\lnot q))$.
    \item Due to the cases above, it suffices to express either $\land$ or $\lor$ using $\lnot,\rightarrow$. $(p \lor q)$ is l.e. to $((\lnot p) \rightarrow  q)$.
\end{enumerate}
\end{proof}

\begin{example}

Some sets of connectives that are not adequate are:
\begin{enumerate}
    \item $\land, \lor$
    \item $\lnot,\leftrightarrow$
\end{enumerate}
\begin{proof}
\begin{enumerate}
    \item If $\phi$ is build using $\land,\lor$ then 
    $F_\phi(T,\dots,T) = T$ as proven by induction over number of connectives.
    \item exercise.
\end{enumerate}
\end{proof}
\end{example}

\begin{example}
The \emph{NOR} connective $\downarrow$ has truth table:
    \begin{tabular}{c|c|c}
        $p$ & $q$ & $(p \downarrow q)$ \\
        \hline
        $T$ & $T$ & $F$ \\ 
        $T$ & $F$ & $F$ \\
        $F$ & $T$ & $F$ \\
        $F$ & $F$ & $T$ \\
        
    \end{tabular}
    It is adequate on its own. (exercise - express $\lnot, \land$)
\end{example}

\subsection{A formal system for propositional logic}

Idea: Try to generate all tautologies from certain basic assumptions (axioms) using appropriate deduction rules.

\begin{definition}
\textbf{This is important!}

A \emph{formal deduction system} $\Sigma$ has the following ingredients:
\begin{enumerate}
    \item An \emph{alphabet} $A$ of symbols $(A \neq \emptyset)$.
    \item A non empty set $\mathcal{J}$ of the set of all finite sequences (`strings`) of the elements of A: the \emph{formulas} of $\Sigma$.
    \item A subset $\mathcal{A} \subseteq \mathcal{J}$ called the \emph{axioms} of $\Sigma$. 
    \item A collection of \emph{deduction rules}.
\end{enumerate}
\end{definition}

\begin{definition}
A \emph{proof} in $\Sigma$ us a finite sequence of formulas in $\mathcal{J}$ \[ \phi_1,\dots,\phi_n \] such that each $ \phi_i $ is either an axiom \emph{or} is obtained from $ \phi_1, \dots, \phi_{i-1} $ using one of the deduction rules. The last (or any) formula in a proof is a \emph{theorem} of $\Sigma$.
Write $\vdash_\Sigma \phi$ for `$\phi$ is a theorem of $\Sigma$`. 
\end{definition}

\begin{remark}
\begin{enumerate}
    \item If $\phi \in \mathcal{A}$ then $\vdash_\Sigma \phi$.
    \item We should have an algorithm to test whether a string of symbols really is a formula and whether it is an axiom. Then someone who is willing to follow an algorithm precisely (computer) should be able to generate all possible proofs in $\sigma$ and check whether something is a proof. (We say $\Sigma$ is \emph{recursive} in this case.)
\end{enumerate}
\end{remark}

\begin{definition}
The formal system $L$ for propositional logic has:

\begin{itemize}
    \item \textbf{Alphabet}: variables $p_1,p_2,p_3\dots$  connectives $\lnot,\rightarrow$
    punctuation ),(.
    \item \textbf{Formulas}: as defined in \ref{Def:formula} and will be called \emph{$L$-formulas}.
    \item \textbf{Axioms}: Suppose $\phi,\psi,\chi$ are $L$-formulas. The following are axioms of $L$:
    \begin{itemize}
        \item[A1] $(\phi \rightarrow (\psi \rightarrow \phi))$
        \item[A2] $((\phi \rightarrow (\psi \rightarrow \chi)) \rightarrow ((\phi \rightarrow \psi) \rightarrow (\psi \rightarrow \chi)))$
        \item[A3]: $(((\lnot \psi) \rightarrow ( \lnot \phi)) \rightarrow (\phi \rightarrow \psi))$
    \end{itemize}
    \item \textbf{deduction rule}: \emph{ Modus Ponens (MP)} from $\phi,  (\phi \rightarrow \psi)$ deduce $\psi$.
\end{itemize}

\begin{example}

Suppose $\phi$ is an $L$-formula. Then $\vdash_L (\phi \rightarrow \phi)$.
A proof in $L$ could be as follows:
\begin{enumerate}
    \item $(\phi \rightarrow ((\phi \rightarrow \phi) \rightarrow \phi))$ use A1
    \item $(\chi \rightarrow (\phi \rightarrow ((\phi \rightarrow \phi) \rightarrow \phi)))$ use A1 and MP
    \item $((\phi \rightarrow (\phi \rightarrow \phi)) \rightarrow (\phi \rightarrow \phi))$
    \item $(\phi \rightarrow (\phi \rightarrow \phi))$
    \item $(\phi \rightarrow \phi)$.
\end{enumerate}
\end{example}
\end{definition}



\end{document}